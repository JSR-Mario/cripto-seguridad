\textbf{Elabora la demostración de que la prueba de divisibilidad entre 9 funciona}\vspace{.2cm}

\textcolor{bibi}{Prueba de divisibilidad}
\begin{quote}
    Primero recordemos la prueba, la idea principal es la siguiente: Si quieres probar que un
    numero n es divisible entre 9, toma los m dígitos de n, digamos $d_0,d_1 \dots d_m$ y los
    sumas $d_0+d_1 +\dots+ d_m=n_1$, $n$ sera divisible entre 9 SII este nuevo número es
    divisible entre 9. \vspace{.3cm}

    \textbf{Ejemplo:}

    $n=455981$, sumamos sus dígitos $4+5+5+9+8+1=32$ que no es divisible entre 9 y por tanto
    $455981$ tampoco lo es. \vspace{.3cm}

    $n=55931112$, sumamos todos sus dígitos $5+5+9+3+1+1+1+2=27$ que si es divisible entre 9 y
    por tanto $55931112$ también lo es.
\end{quote}

\textcolor{bibi}{Demostración}
\begin{quote}
    Intente por inducción pero no me salio así que tuve que buscar una parte :c\dots

    Una parte muy importante, es ver que estos dígitos no están sueltos, es decir, estos
    describen a $n$ usando la siguiente ec:
    \begin{align*}
        n=d_m 10^m + d_{m-1} 10^{m-1} + \dots + d_1 10 + d_0 
    \end{align*}

    Ademas, vamos a usar congruencias; anotando lo siguiente: $a \equiv b \ (mod \ m)$ significa
    m divide $a-b$, de aquí tenemos que $10 \equiv 1 \ (mod \ 9)$ y de la misma manera $10^i
    \equiv 1^i \equiv 1 \ (mod \ 9)$. Ahora solo multiplicamos por el dígito:
    \begin{align*}
        d_i 10^i \equiv d_i 1 \equiv d_i \ (mod \ 9)
    \end{align*}

    Ahora lo hacemos para todos los dígitos de $n$:
    \begin{align*}
        n = \displaystyle\sum_{i=0}^{m} d_i 10^i \equiv \displaystyle\sum_{i=0}^{m} d_i \
        (mod \ 9)
    \end{align*}

    Pero nota como la segunda parte de la ec. es la suma de los dígitos. Si uno es divisible
    por 9 entonces el otro lo es. 
\end{quote}
