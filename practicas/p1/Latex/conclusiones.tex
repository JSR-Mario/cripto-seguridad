Para cerrar, queremos expresar como equipo lo que observamos durante la P1. Lo primero que consideramos
relevante destacar es lo interesante y casi sorprendente que resultó transformar archivos que parecían
ruido o errores en videos, PDFs, entre otros formatos. Esto evidencia lo potente que puede ser el entorno
de la criptografía. Ahora bien, algo que nos llamó la atención fue que cuatro estudiantes sin una
preparación profunda fuimos capaces de romper todos estos métodos de cifrado. Esto nos lleva a concluir
que, en el contexto actual, tales métodos resultan demasiado vulnerables: son simples y pueden ser
quebrados con relativa facilidad mediante fuerza bruta o análisis básico. Nos interesa continuar con el
curso para aprender cómo expandir estos algoritmos y diseñar propuestas más robustas a partir de
herramientas similares. \vspace{.3cm}

Otro aspecto que identificamos fue la dificultad de trabajar en equipo a distancia, sin conocernos
previamente y sin tener experiencia sólida en la materia. Aun así, logramos llegar a un producto que,
aunque no fue exactamente lo que esperábamos, representó un esfuerzo conjunto. Además, consideramos que
hubiera sido útil plantear más dudas sobre las especificaciones de la práctica con varios días de
anticipación, lo que nos habría permitido organizarnos de una manera más eficiente. \vspace{.3cm}

Como aprendizaje, nos llevamos que aún queda mucho por explorar en el campo de la criptografía y que
estamos motivados para realizar una entrega más sólida en la siguiente ocasión.
