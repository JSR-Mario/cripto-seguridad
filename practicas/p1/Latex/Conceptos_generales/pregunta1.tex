\textbf{¿Cuantos primos relativos hay en $\mathbb{Z}_{256}$?} \vspace{.3cm}

\begin{quote}
     Para responder esta pregunta voy a hacer uso de la
     función $\phi$ de Euler \citep{wikipedia_euler_totient_es}. La función $\phi (n) $ cuando
     $n=256=2^8$ cuenta la cantidad de enteros positivos menores o iguales a n que son
     primos relativos con este mismo:

     \begin{align*}
         \phi(p^k) &= p^k - p^{k-1} \\
         &= 2^8 - 2^7 \\
         &= 256 - 128 \\
         &= 128 \\
     \end{align*}

     Por tanto hay 128 enteros en el anillo que son primos relativos a $256$.
\end{quote}
\vspace{.5cm}
