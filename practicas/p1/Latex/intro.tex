La siguiente práctica tiene como objetivo aprender cómo funcionan los cifrados clásicos y programar
de manera practica un descifrador de:

\begin{itemize}
    \item \textbf{Cifrado César}
    \item \textbf{Cifrado decimado}
    \item \textbf{Cifrado afín}
    \item \textbf{Base64}
\end{itemize}

Ademas busca explorar las firmas de los archivos para dar entendimiento de que son los
\texttt{Magic Bytes} y su importancia a la hora de poder $leer$ un archivo. Durante el
desarollo, se utilizan 2 LDP para leer/manipular y escribir bytes. \vspace{.3cm}

El desarrollo de la practica es importante pues nos dará pie a ver como funcionan estos
fundamentos teórico matemáticos a la hora de hacer encripciones que sean practicas y duras.
Ademas nos brindara familiaridad con los sistemas de archivos y la manipulacion a nivel de bytes.
\vspace{.3cm}
