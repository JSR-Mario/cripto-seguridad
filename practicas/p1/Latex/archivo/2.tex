\textbf{Descifrar el archivo file2.lol}
\begin{quote}
    Al abordar el archivo \texttt{file2.lol}, se veían sus primeros bytes
    en ASCII y en hexadecimal. El contenido no presentaba texto comprensible, por lo que se
    asumió que se trataba de un cifrado clásico aplicado \emph{byte a byte} en $Z_{256}$.

    Para automatizar el proceso, se hizo un programa que, a partir del encabezado,
    prueba sistemáticamente los cifrados \textbf{César} y \textbf{Afín} sobre los \emph{magic bytes}
    de varios formatos (PDF, PNG, MP3, MP4). La estrategia consiste en buscar un
    desplazamiento (en el caso de César) o un par $(a,b)$ (en el caso afín) que haga
    coincidir el encabezado descifrado con las firmas típicas de archivo.

    En particular, para \textbf{César} se verifica si existe un \emph{shift} constante $s$ tal que
    para todos los bytes del encabezado se cumpla:
    \[
      c_i \equiv p_i + s \pmod{256},
    \]
    donde $p_i$ son los bytes de la firma conocida (por ejemplo, PNG) y $c_i$ los bytes
    observados en el archivo cifrado. Si ese mismo $s$ explica todos los bytes probados,
    se considera una detección positiva y se procede a descifrar con:
    \[
      p_i \equiv c_i - s \pmod{256}.
    \]


    
    Finalmente, se guardó el binario descifrado con extensión \texttt{.png} y se verificó su
    apertura correcta con un visor de imágenes, validando el éxito del proceso.
\end{quote}
\vspace{.5cm}